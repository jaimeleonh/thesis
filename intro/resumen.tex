\documentclass[../main.tex]{subfiles}
\begin{document}
\thispagestyle{plain}
\begin{center}
\textbf{Resumen}
\end{center}
\addcontentsline{toc}{chapter}{Resumen}
\markboth{RESUMEN}{}

Esta tesis presenta una búsqueda de producción de pares de bosones de Higgs en el \textit{Large Hadron Collider} (LHC). Este proceso está predicho por el Modelo Estándar de física de partículas (ME), con una sección eficaz de producción alrededor de 1000 veces más pequeña que la producción de un único bosón de Higgs. El estudio de este proceso aporta información acerca del potencial del Higgs, el autoacoplo del bosón de Higgs y su acoplo a otras partículas del ME, como el quark t or los bosones vectoriales. Adicionalmente, efectos de Física Más Allá del Modelo Estándar se pueden manifestar como modificaciones en la sección eficaz de producción o en la cinemática de los objetos del evento. Dos modos de producción son estudiados: fusión de gluones (\textit{gluon fusion}, ggF) y de bosones vectoriales (\textit{vector boson fusion}, VBF). Esta búsqueda se ha realizado con datos del experimento \textit{Compact Muon Solenoid} (CMS) durante el \textit{Run} 2 del LHC, correspondiente a los años 2016, 2017 y 2018, correspondiente a una luminosidad integrada de 138~fb${}^{-1}$. Este análisis considera el estado final en el que un bosón de Higgs decae en dos quarks b y el otro en dos leptones $\tau$. Dado que es este canal tiene una fracción de desintegración intermedia y un fondo relativamente bajo, es uno de los canales que provee mejores resultados dentro de la producción de pares de bosones de Higgs.

En primer lugar se presenta el contexto teórico de la producción de bosones de Higgs, seguida de una descripción de la configuración experimental usada en el análisis, incluyendo las técnicas usadas en el experimento CMS para la reconstrucción e identificación de partículas. A continuación se describe la estrategia seguida en el análisis, comenzando por una descripción de la selección aplicada a los diferentes objectos del estado final de cara a reducir la contaminación del fondo. Una mayor discriminación se produce separando los eventos de datos y simulación en categorías enriquecidas en eventos producidos por ggF o VBF. Para la categorización en eventos de VBF se consideran técnicas de aprendizaje automático, concretamente una red neuronal multiclase que clasifica los eventos como procedentes de señal o de alguno de los fondos. La extracción de la señal se realiza con otra red neuronal, la cual provee los principales resultados físicos de esta tesis: límites superiores en la sección eficaz de producción de pares de bosones de Higgs mediante ggF o VBF y en la sección eficaz de producción solo mediante VBF (asumiendo que ggF se comporta como su preddición en el ME), y rangos de exclusión para el autoacoplo del bosón de Higgs y su acoplo a los bosones vectoriales. Los resultados obtenidos constituyen una importante mejora con respecto al anterior análisis \hhbbtt{}, obteniendo una de las mejores sensibilidades entre los canales de desintegración estudiados dentro de la colaboración CMS. De cara al próximo periodo de toma de datos (\textit{Run} 3, 2022-2025), se ha estudiado una nueva estrategia de selección \textit{online}, apuntando al canal en el que ambos $\tau$ se desintegran en hadrones y adicionalmente un \textit{jet} se encuentra en el estado final. Esta nueva estrategia ha sido incluida en los menús de \textit{trigger} antes del periodo de toma de datos que comenzó en 2022, de forma que su rendimiento puede ser estudiado usando datos de colisiones reales y simuladas.

De cara a preparar el siguiente periodo de operaciones del LHC que comienza en 2029, el llamado \textit{High-Luminosity} LHC (HL-LHC), el experimento CMS está llevando a cabo una actualización del detector para que pueda trabajar con el aumento de luminosidad previsto, el cual llevará asociado un aumento de ocupancia y \textit{rate} de datos. Se ha desarrollado un nuevo algoritmo para el sistema de disparo para realizar la generación de \textit{trigger primitives} en las cámaras de tubos de deriva (DT) durante HL-LHC. Este algoritmo se beneficia de la resolución temporal completa que proveerá la nueva electrónica directamente a nivel del sistema de disparo, calculando el tiempo, posición y dirección de la \textit{trigger primitive} con una resolución comparable a la que obtiene actualmente el sistema de reconstrucción \textit{offline}. El rendimiento de este algoritmo ha sido estimado con un emulador en \textit{software} en muestras de datos reales y simulados y con tests para la implementación en \textit{hardware}.


\newpage
\end{document}
