\documentclass[../main.tex]{subfiles}
\begin{document}
\chapter*{Introduction}
\markboth{INTRODUCTION}{}
\addcontentsline{toc}{chapter}{Introduction}

The Standard Model of particle physics is a theory that describes the elementary particles and their interactions. The last piece of this model was included in 1964 with the introduction of the electroweak symmetry breaking and the Higgs boson. Since then, the Standard Model has provided many successful prediction experimentally verified thanks to, for instance, collider experiments such as the LHC. In fact, the ATLAS and CMS experiment using proton-proton collision data from the LHC were able to discover the Higgs boson in 2012 \cite{intro:theo:cms_higgs, intro:theo:atlas_higgs}, which strongly supports the validity of the Standard Model.

Nonetheless, some phenomena are not described by the Standard Model, such as the asymmetry between matter and antimatter, the existence of dark matter, gravity, or even the mass scale of the Higgs boson itself. Therefore, numerous theories that extend the Standard Model at higher energies have been proposed, so they can explain the so-called Beyond the Standard Model physics.

The production of Higgs boson pairs is a process that can help to further understand the Standard Model and test the validity of Beyond the Standard Model theories. It provides unique information about the Higgs self-coupling and its coupling to other particles. Deviations on their Standard Model predictions can be related to Beyond the Standard Model physics processes.

The observation of Higgs boson pair production (as predicted by the Standard Model) and the determination of the Higgs self-coupling are not possible with the data collected as of now by the LHC experiments. The study of some HH decay channels, however, can provide very stringent upper limits on its production cross section and exclusion ranges for the self-coupling already with LHC Run 2 data. Among these decay channels, the one were one Higgs boson decays into two b quarks and the other into two $\tau$ leptons gives some of the most significant results, thanks to its medium branching fraction and relatively low background.

Looking into the future, the High-Luminosity LHC era will continue the current LHC operations, providing much more instantaneous luminosity and collected data. This amount of new data will open the door to the study of rarer processes, or even provide more precise results for the already studied ones (such as the Higgs boson pair production). This accelerator upgrades, however, need to be addressed by the experiments, so they can cope with the increased data rate and radiation. Within the CMS experiment, various subdetectors will be replaced, upgraded or substituted, so the final trigger and reconstruction performance will be maintained or even improved during HL-LHC operation.

This thesis is structured as follows. Chapter~\ref{intro:chap:theo} introduces the theoretical background needed in order to study the Higgs boson pair production at the LHC. Chapter~\ref{intro:chap:exp} describes the experimental setup used in the analysis, including the LHC and the CMS experiment. Chapter~\ref{intro:sec:id} presents the particle reconstruction and identification methods used in CMS analyses. The \hhbbtt{} analysis is discussed in Chapter~\ref{hh:chapter:analysis}, and its results are shown in Chapter~\ref{hh:chapter:results}. Chapter~\ref{hh:chapter:trigger} discusses a new trigger strategy that could boost the sensitivity in the \hhbbtt{} and \htt{} analyses during Run 3. Finally, Chapter~\ref{dts:chapter:intro} describes a new algorithm implemented for the CMS Drift Tube chambers for HL-LHC and shows its performance results over simulated and collision data samples.



\end{document}

