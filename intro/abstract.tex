\documentclass[../main.tex]{subfiles}
\begin{document}
\thispagestyle{plain}
\begin{center}
\textbf{Abstract}
\end{center}
\addcontentsline{toc}{chapter}{Abstract}
\markboth{ABSTRACT}{}

This thesis reports a search for double Higgs boson production at the Large Hadron Collider (LHC). This process is predicted by the Standard Model of particle physics (SM), with a production cross section around 1000 times smaller than single Higgs boson production. Its study can provide information about the Higgs potential, the Higgs self-coupling, and its couplings to other SM particles such as the t quark or the vector bosons. Additionally, Beyond the Standard Model (BSM) effects can manifest as modifications on the production cross section or changes in event kinematics. Two production modes are studied, gluon fusion (ggF) and vector boson fusion (VBF). This search is performed using data from the Compact Muon Solenoid (CMS) experiment during the so-called Run 2 of the LHC, corresponding to proton-proton collision data collected in 2016, 2017, and 2018, and amounting to a total integrated luminosity of around 138~fb${}^{-1}$. The analysis considers the final state where a Higgs boson decays into two b quarks and the other decays into two $\tau$ leptons. Since it has an intermediate branching ratio and relatively low background, it is considered as one of the golden channels in the search for double Higgs boson production. 

The theoretical context of the double Higgs production is first presented, followed by a description of the experimental setup used in the analysis, including the techniques used within the CMS experiment for particle reconstruction and identification. The analysis strategy is then presented, starting by identifying the different background and signal samples to be considered and followed by a description of the selection applied to the different objects in the final state in order to reduce the background contamination. A further discrimination is then performed by splitting the data and simulated events into categories enriched in either ggF or VBF events. For the VBF categorisation a machine learning approach is considered, consisting in a multi-class deep neural network (DNN) that classifies events as signal or as one of the main background processes. Signal extraction is finally performed with another DNN. The main physics results of this thesis are: upper limits on the double Higgs boson production cross section via ggF and VBF, the production cross section for VBF production only (fixing ggF to its SM prediction), and exclusion ranges on the Higgs self-coupling and its couplings to vector bosons. The results obtained constitute an important improvement with respect to the previous \hhbbtt{} analysis, providing one of the best sensitivities among all the decay channels studied in the CMS Collaboration. For the new data taking period (Run 3, 2022-2025), a new trigger strategy has been studied, targeting the decay channel where both $\tau$ decay into hadrons and a jet is present in the final state. This new trigger strategy has been included in the trigger menus before the data-taking period starting in 2022, so its performance can be studied using real and simulated collision data. 

In order to prepare for the next stage of LHC operations after 2029, the so-called High-Luminosity LHC (HL-LHC) era, the increase of luminosity, resulting in a larger occupancy and trigger rate, will require a CMS detector upgrade. A new trigger algorithm has been developed to perform the trigger primitive generation in the muon drift tube chambers during HL-LHC. This algorithm profits from the full time resolution provided by the new electronics at trigger level, computing the trigger primitive's time, position, and direction with a resolution comparable to what is reachable with the present offline reconstruction system. The algorithm's performance is estimated using a software emulator with simulated and real data samples, and through hardware implementation tests.

\newpage
\end{document}

